\documentclass[]{elsarticle} %review=doublespace preprint=single 5p=2 column
%%% Begin My package additions %%%%%%%%%%%%%%%%%%%
\usepackage[hyphens]{url}

  \journal{arXiv.org} % Sets Journal name


\usepackage{lineno} % add
\providecommand{\tightlist}{%
  \setlength{\itemsep}{0pt}\setlength{\parskip}{0pt}}

\usepackage{graphicx}
\usepackage{booktabs} % book-quality tables
%%%%%%%%%%%%%%%% end my additions to header

\usepackage[T1]{fontenc}
\usepackage{lmodern}
\usepackage{amssymb,amsmath}
\usepackage{ifxetex,ifluatex}
\usepackage{fixltx2e} % provides \textsubscript
% use upquote if available, for straight quotes in verbatim environments
\IfFileExists{upquote.sty}{\usepackage{upquote}}{}
\ifnum 0\ifxetex 1\fi\ifluatex 1\fi=0 % if pdftex
  \usepackage[utf8]{inputenc}
\else % if luatex or xelatex
  \usepackage{fontspec}
  \ifxetex
    \usepackage{xltxtra,xunicode}
  \fi
  \defaultfontfeatures{Mapping=tex-text,Scale=MatchLowercase}
  \newcommand{\euro}{€}
\fi
% use microtype if available
\IfFileExists{microtype.sty}{\usepackage{microtype}}{}
\usepackage[margin=1.0in]{geometry}
\bibliographystyle{elsarticle-harv}
\ifxetex
  \usepackage[setpagesize=false, % page size defined by xetex
              unicode=false, % unicode breaks when used with xetex
              xetex]{hyperref}
\else
  \usepackage[unicode=true]{hyperref}
\fi
\hypersetup{breaklinks=true,
            bookmarks=true,
            pdfauthor={},
            pdftitle={Linear interaction of treatment with baseline risk was sufficient for predicting individualized benefit},
            colorlinks=false,
            urlcolor=blue,
            linkcolor=magenta,
            pdfborder={0 0 0}}
\urlstyle{same}  % don't use monospace font for urls

\setcounter{secnumdepth}{5}
% Pandoc toggle for numbering sections (defaults to be off)

% Pandoc citation processing
\newlength{\cslhangindent}
\setlength{\cslhangindent}{1.5em}
\newlength{\csllabelwidth}
\setlength{\csllabelwidth}{3em}
% for Pandoc 2.8 to 2.10.1
\newenvironment{cslreferences}%
  {}%
  {\par}
% For Pandoc 2.11+
\newenvironment{CSLReferences}[2] % #1 hanging-ident, #2 entry spacing
 {% don't indent paragraphs
  \setlength{\parindent}{0pt}
  % turn on hanging indent if param 1 is 1
  \ifodd #1 \everypar{\setlength{\hangindent}{\cslhangindent}}\ignorespaces\fi
  % set entry spacing
  \ifnum #2 > 0
  \setlength{\parskip}{#2\baselineskip}
  \fi
 }%
 {}
\usepackage{calc}
\newcommand{\CSLBlock}[1]{#1\hfill\break}
\newcommand{\CSLLeftMargin}[1]{\parbox[t]{\csllabelwidth}{#1}}
\newcommand{\CSLRightInline}[1]{\parbox[t]{\linewidth - \csllabelwidth}{#1}\break}
\newcommand{\CSLIndent}[1]{\hspace{\cslhangindent}#1}

% Pandoc header
\renewcommand*\familydefault{\sfdefault}
\usepackage{setspace}
\usepackage{amsmath}
\doublespacing
\usepackage{amssymb}
\usepackage{bm}
\usepackage{caption}
\usepackage{booktabs}
\date{}
\newcommand\given[1][]{\:#1\vert\:}
\usepackage{booktabs}
\usepackage{longtable}
\usepackage{array}
\usepackage{multirow}
\usepackage{wrapfig}
\usepackage{float}
\usepackage{colortbl}
\usepackage{pdflscape}
\usepackage{tabu}
\usepackage{threeparttable}
\usepackage{threeparttablex}
\usepackage[normalem]{ulem}
\usepackage{makecell}
\usepackage{xcolor}



\begin{document}
\begin{frontmatter}

  \title{Linear interaction of treatment with baseline risk was
sufficient for predicting individualized benefit}
    \author[1]{Alexandros Rekkas}
  
    \author[1]{Peter R. Rijnbeek}
  
    \author[2]{Ewout W. Steyerberg}
  
    \author[3]{David van Klaveren}
  
      \address[1]{Department of Medical Informatics, Erasmus Medical
Center, Rotterdam, The Netherlands}
    \address[2]{Department of Biomedical Data Sciences, Leiden
University Medical Center, Leiden, The Netherlands}
    \address[3]{Department of Public Health, Erasmus Medical Center,
Rotterdam, The Netherlands}
    
  \begin{abstract}
  \textbf{Objective}: To compare different risk-based methods predicting
  individualized treatment effects in RCTs. \textbf{Study Design and
  Setting}: We simulated RCT data using diverse assumptions for the
  average treatment effect, a baseline prognostic index of risk (PI),
  the shape of its interaction with treatment (none, linear, quadratic
  or non-monotonic) and the magnitude of treatment-related harms. In
  each sample we predicted absolute benefit using: models with a
  constant relative treatment effect; stratification in quarters of the
  PI; models including a linear interaction of treatment with the PI;
  models including an interaction of treatment with a restricted cubic
  spline (RCS) transformation of the PI; an adaptive approach using
  Akaike's Information Criterion. We evaluated predictive performance
  using root mean squared error and measures of discrimination and
  calibration for benefit. \textbf{Results}: The linear-interaction
  model and the RCS-interaction model outperformed the constant
  treatment effect model in many simulation scenarios. The RCS-model was
  optimal when quadratic or non-monotonic deviations from a constant
  treatment effect were stronger, and when sample size was larger.
  Larger sample size also supported the adaptive approach.
  \textbf{Conclusion}: An interaction between risk and treatment
  assignment generally improved treatment effect predictions. Non-linear
  RCS interactions should be considered for larger sample size.
  \end{abstract}
   \begin{keyword} treatment effect heterogeneity absolute
benefit prediction models\end{keyword}
 \end{frontmatter}

\doublespacing 
\linenumbers

\hypertarget{introduction}{%
\section{Introduction}\label{introduction}}

Predictive approaches for assessing heterogeneity of treatment effects
(HTE) aim at the development of models predicting either individualized
effects or which of two (or more) treatments is better for an individual
{[}1{]}. In prior work, we divided such methods in three broader
categories based on the reference class used for defining patient
similarity when making individualized predictions or recommendations
{[}2{]}. Risk-modeling approaches use prediction of baseline risk as the
reference; treatment effect modeling approaches also model
treatment-covariate interactions, in addition to risk factors; optimal
treatment regime approaches focus on developing treatment assignment
rules and therefore rely heavily on modeling treatment effect modifiers.

Risk-modeling approaches to predictive HTE analyses provide a viable
option in the absence of well-established treatment effect modifiers
{[}3,4{]}. In simulations, modeling of effect modifiers, i.e.
treatment-covariate interactions, often led to miscalibrated predictions
of benefit, while risk-based methods proved quite robust {[}5{]}. Most
often, risk-modeling approaches are carried out in two steps: first a
risk prediction model is developed externally or internally on the
entire RCT population, ``blinded'' to treatment; then the RCT population
is stratified using this prediction model to evaluate risk-based
treatment effect variation {[}6{]}. This two-step approach identified
substantial absolute treatment effect differences between low-risk and
high-risk patients in a re-analysis of 32 large trials {[}7{]}. However,
even though estimates at the risk subgroup level may be accurate, these
estimates do not apply to individual patients, especially for patients
with predicted risk at the boundaries of the risk intervals. Hence, the
risk-stratified approach is useful for exploring and presenting HTE, but
is not useful for supporting treatment decisions for individual
patients.

To individualize treatment effects, the recent PATH statement suggested
various risk-based models including a prognostic index of baseline risk
(PI) and treatment assignment {[}3,4{]}. We aimed to summarize and
compare different risk-based models for predicting individualized
treatment effects. We simulated RCT settings to compare the performance
of these models under different assumptions of the relationship between
baseline risk and treatment. We illustrated the different models by a
case study of predicting individualized effects of tissue plasminogen
activator (tPA) versus streptokinase treatment in patients with an acute
myocardial infarction (MI).

\hypertarget{methods}{%
\section{Methods}\label{methods}}

\hypertarget{simulation-scenarios}{%
\subsection{Simulation scenarios}\label{simulation-scenarios}}

We simulated a typical RCT that is undertaken to compare a binary
outcome (e.g.~death) between a group of treated patients in the
treatment arm and a group of untreated patients in the control arm. For
each patient we generated 8 baseline covariates
\(x_1,\dots,x_4\sim N(0, 1)\) and \(x_5,\dots,x_8\sim B(1, 0.2)\).
Treatment was allocated using a 50:50 split. Outcomes for patients in
the control arm were generated from a logistic regression model
including all baseline covariates. In the base scenarios coefficient
values were such, that the AUC of the logistic regression model was
\(0.75\) and the event rate in the control arm was \(20\%\). Binary
outcomes in the control arm were generated from Bernoulli variables with
true probabilities
\(P(y=1|X, t_x = 0) = \text{expit}(PI)=\frac{e^{PI}}{1+e^{PI}}\).

Outcomes in the treatment arm were generated using 3 base scenarios:
absent treatment effect (OR = 1), moderate treatment effect (OR = 0.8)
and strong treatment effect (OR = 0.5). We started with simulating
outcomes based on true constant relative treatment effects for the 3
base scenarios. We then simulated linear, quadratic and non-monotonic
deviations from constant treatment effects using:
\[lp_1 = \gamma_2(PI-c)^2 + \gamma_1(PI-c) + \gamma_0, \] where \(lp_1\)
is the true linear predictor in the treatment arm, so that
\(P(y=1|X, t_x=1) = \text{expit}(lp_1)\). Finally, we simulated
scenarios where a constant absolute harm is applied across all treated
patients. In this case we have
\(P(y=1|X,t_x=1) = \text{expit}(lp_1) + \text{harm}\).

The sample size for the base scenarios was set to 4,250 (\(80\%\) power
for the detection of a marginal OR of 0.8). We evaluated the effect of
smaller or larger sample sizes of 1,063 (4,250 divided by 4) and 17,000
(4250 multiplied by 4), respectively. We also evaluated the effect of
worse or better discriminative ability for risk, adjusting the baseline
covariate coefficients, such that the AUC of the regression model in the
control arm was 0.65 and 0.85 respectively.

Combining all these settings resulted in a simulation study of 648
scenarios (exact settings in the supplementary material). With these
scenarios we were able to cover the observed treatment effect
heterogeneity in 32 large trials as well as many other potential
variations of risk-based treatment effect {[}7{]}.

\hypertarget{individualized-risk-based-benefit-predictions}{%
\subsection{Individualized risk-based benefit
predictions}\label{individualized-risk-based-benefit-predictions}}

All risk-based methods assume that a risk prediction model is available
to assign risk predictions to individual patients. For the simulations
we developed a prediction model internally, using logistic regression
including main effects for all baseline covariates and treatment
assignment. Risk predictions for individual patients were based on
treatment assignment to the control arm, that is setting treatment
assignment to 0.

A \emph{stratified HTE method} has been suggested as an alternative to
traditional subgroup analyses. Patients are stratified into
equally-sized risk strata---in this case based on risk quartiles.
Absolute treatment effects within risk strata are estimated by the
difference in event rate between patients in the control arm and
patients in the treated arm. We considered this approach as a reference,
expecting it to perform worse than the other candidates, as its
objective is not to individualize benefit prediction.

Second, we considered a model which assumes \emph{constant relative
treatment effect} (constant odds ratio). Hence, absolute benefit is
predicted from
\(\hat{\tau}(\bm{x}) = \text{expit}(PI +\log(\text{OR}))\).

Third, we considered a logistic regression model including treatment,
the prognostic index, and their linear interaction. Absolute benefit is
then estimated from
\(\hat{\tau}(\bm{x})=\text{expit}(\beta_0+\beta_{PI}PI) - \text{expit}(\beta_0+\beta_{t_x}+(\beta_{PI}+\beta_*)PI)\).
We will refer to this method as the \emph{linear interaction} approach.

Fourth, we used \emph{restricted cubic splines} (RCS) to relax the
linearity assumption on the effect of the linear predictor {[}8{]}. We
considered splines with 3 (RCS-3), 4 (RCS-4) and 5 (RCS-5) knots to
compare models with different levels of flexibility.

Finally, we considered an adaptive approach using Akaike's Information
Criterion (AIC) for model selection. The candidate models were: a
constant treatment effect model, a model with a linear interaction with
treatment and RCS models with 3, 4 and 5 knots.

\hypertarget{evaluation-metrics}{%
\subsection{Evaluation metrics}\label{evaluation-metrics}}

We evaluated the predictive accuracy of the considered methods by the
root mean squared error (RMSE):

\[\text{RMSE}=\sqrt{\frac{1}{n}\sum_{i=1}^n\big(\tau(\bm{x}_i) - \hat{\tau}(\bm{x}_i)\big)^2}\]
We compared the discriminative ability of the methods under study using
c-for-benefit {[}9{]}. The c-for-benefit represents the probability that
from two randomly chosen matched patient pairs with unequal observed
benefit, the pair with greater observed benefit also has a higher
predicted benefit. To be able to calculate observed benefit, patients in
each treatment arm are ranked based on their predicted benefit and then
matched 1:1 across treatment arms. \emph{Observed} treatment benefit is
defined as the difference of observed outcomes between the untreated and
the treated patient of each matched patient pair. \emph{Predicted}
benefit is defined as the average of predicted benefit within each
matched patient pair.

We evaluated calibration in a similar manner, using the integrated
calibration index (ICI) for benefit {[}10{]}. The observed benefits are
regressed on the predicted benefits using a locally weighted scatterplot
smoother (loess). The ICI-for-benefit is the average absolute difference
between predicted and smooth observed benefit. Values closer to \(0\)
represent better calibration.

\hypertarget{results}{%
\section{Results}\label{results}}

\hypertarget{simulations}{%
\subsection{Simulations}\label{simulations}}

The linear interaction model outperformed all RCS methods in terms of
RMSE in scenarios with true constant relative treatment effect (OR =
0.8, N = 4,250 and AUC = 0.75), strong linear and even strong quadratic
deviations from a constant relative treatment effect (Figure
\ref{fig:rmsebase}; panels A-C). However, with non-monotonic deviations
from a constant relative treatment effect, the RMSE of the linear
interaction model increased substantially, especially in the presence of
treatment-related harms (Figure \ref{fig:rmsebase}; panel D). In these
scenarios, RCS-3 outperformed all other methods in terms of RMSE. The
constant treatment effect approach had overall best performance under
true constant treatment effect settings, but was sensitive to all
considered deviations, resulting in increased RMSE. Finally, the
adaptive approach had comparable performance to the best-performing
method in each scenario. However, in comparison with the best-performing
approach, its RMSE was more variable across the 500 replications in the
scenarios with linear and non-monotonic deviations, especially when also
including moderate or strong treatment-related harms. This is caused by
wrongly selecting the constant treatment effect model in a substantial
proportion of the replications (Supplement, Figure S1).

\begin{figure}
\includegraphics[width=1\linewidth]{manuscript_files/figure-latex/rmsebase-1} \caption{RMSE of the considered methods across 500 replications calculated in a simulated super-population of size 500,000. The scenario with true constant relative treatment effect (panel A) had a true prediction AUC of 0.75 and sample size of 4250. The RMSE is also presented for strong linear (panel B), strong quadratic (panel C), and non-monotonic (panel D) from constant relative treatment effects. Panels on the right side present the true relationship between baseline risk (x-axis) and absolute treatment benefit (y-axis). The 2.5, 25, 50, 75, and 97.5 percentiles of the risk distribution are expressed by the boxplot.}\label{fig:rmsebase}
\end{figure}

Increasing the sample size to 17,000 favored RCS-3 the most, as it
achieved lowest or close to lowest RMSE across all scenarios (Figure
\ref{fig:rmsesamplesize}). Especially in cases of strong quadratic and
non-monotonic deviations RCS-3 had lower RMSE (median 0.011 for strong
quadratic deviations and 0.010 for non-monotonic deviations with no
treatment-related harms) compared to the linear interaction approach
(median 0.013 and 0.014, respectively), regardless of the strength of
treatment-related harms. Due to the large sample size, the RMSE of the
adaptive approach was even more similar to the best-performing method,
and the constant relative treatment effect model was less often wrongly
selected (Supplement, Figure S2).

\begin{figure}
\includegraphics[width=1\linewidth]{manuscript_files/figure-latex/rmsesamplesize-1} \caption{RMSE of the considered methods across 500 replications calculated in a simulated sample of size 500,000. Sample size 17,000 rather than 4250 in Figure \ref{fig:rmsebase}}\label{fig:rmsesamplesize}
\end{figure}

When we increased the AUC of the true prediction model to 0.85 (OR = 0.8
and N = 4,250). RCS-3 had the lowest RMSE in the case of strong
quadratic or non-monotonic deviations and very comparable performance to
the -- optimal -- linear interaction model in the case of strong linear
deviations (median RMSE 0.016 for RCS-3 compared to 0.014 for the linear
interaction model). As observed in the base case scenario the adaptive
approach wrongly selected the constant treatment effect model (23\% and
25\% of the replications in the strong linear and non-monotonic
deviation scenarios without treatment-related harms, respectively),
leading to more variability of the RMSE (Supplement, Figure S3).

\begin{figure}
\includegraphics[width=1\linewidth]{manuscript_files/figure-latex/rmseauc-1} \caption{RMSE of the considered methods across 500 replications calculated in a simulated sample of size 500,000. True prediction AUC of 0.85 and sample size of 4,250.}\label{fig:rmseauc}
\end{figure}

In comparison with the true approach, dsicrimination for benefit in the
scenario with a constant relative treatment effect was only slightly
lower for the linear interaction model, but more substantially lower for
the non-linear RCS approaches (Figure \ref{fig:discrimination}; panel
A). With strong linear or quadratic deviations from a constant relative
treatment effect, all methods discriminated quite similarly (Figure
\ref{fig:discrimination}; panels B-C). In the scenario with
non-monotonic deviations, the constant effect model had much lower
discriminative ability compared to non-linear RCS methods (median AUC of
0.4971 for the constant effects model, 0.5285 for the linear interaction
model and 0.5304 for the best-performing RCS-3; Figure
\ref{fig:rmseauc}; panel D). With increasing number of RCS knots, we
observed decreasing median values and increasing variability of the
c-for-benefit in all scenarios. \textbf{\emph{With increasing sample
size\ldots{} (Suppl Fig\ldots)}}

\begin{figure}
\includegraphics[width=1\linewidth]{manuscript_files/figure-latex/discrimination-1} \caption{Discrimination for benefit of the considered methods across 500 replications calculated in a simulated sample of size 500,000. True prediction AUC of 0.75 and sample size of 4,250.}\label{fig:discrimination}
\end{figure}

In terms of calibration for benefit, the constant effects model
outperformed all other models in the scenario with true constant
treatment effects, but was miscalibrated for all deviation scenarios
(Figure \ref{fig:calibration}). The linear interaction model showed best
or close to best calibration across all scenarios and only showed worse
calibration compared to RCS-3 in case of non-monotonic deviations and
treatment-related harms (Figure \ref{fig:calibration}; panel D). The
adaptive approach was worse calibrated in scenarios with strong linear
and non-monotonic deviations compared to the linear interaction model
and RCS-3. \textbf{\emph{With increasing sample size\ldots{} (Supp
Fig\ldots)}}

The results from all individual scenarios can be explored online at
\url{https://arekkas.shinyapps.io/simulation_viewer/}.

\begin{figure}
\includegraphics[width=1\linewidth]{manuscript_files/figure-latex/calibration-1} \caption{Calibration for benefit of the considered methods across 500 replications calculated in a simulated sample of size 500,000. True prediction AUC of 0.75 and sample size of 4,250.}\label{fig:calibration}
\end{figure}

\hypertarget{case-study}{%
\subsection{Case study}\label{case-study}}

We demonstrate the different methods for individualizing treatment
benefits using data from 30,510 patients with an acute myocardial
infarction (MI) included in the GUSTO-I trial. 10,348 patients were
randomized to tissue plasminogen activator (tPA) treatment and 20,162
were randomized to streptokinase. The outcome of interest was 30-day
mortality, recorded for all patients.

In line with previous analyses {[}11,12{]}, we fitted a logistic
regression model with 6 baseline covariates, i.e.~age, Killip class,
systolic blood pressure, heart rate, an indicator of previous MI, and
the location of MI, to predict 30-day mortality risk. A constant effect
of treatment was included in the model. When deriving risk predictions
for individuals we set the treatment indicator to 0. More information on
model development can be found in the supplement.

We used the risk linear predictor to fit the proposed methods under
study for individualizing absolute benefit predictions. All methods
predicted increasing benefits for patients with higher baseline risk
predictions, but the fitted patterns were clearly different. The
adaptive approach selected the model with RCS smoothing with 4 knots.
However, for very low baseline risk the decreasing predicted benefit
with increasing risk may be somewhat too flexible. The more robust
models, the linear interaction model or the model with RCS smoothing (3
knots), gave very similar benefit predictions, followed the evolution of
the stratified estimates very closely and may therefore be preferable
for use in clinical practice. The linear interaction model had somewhat
lower AIC compared to the model with RCS smoothing (3 knots), slightly
better cross-validated discrimination (c-for-benefit 0.526 vs 0.525) and
quite similar cross-validated calibration (ICI-for benefit 0.0115 vs
0.0117).

\begin{figure}
\includegraphics[width=1\linewidth]{manuscript_files/figure-latex/gusto-1} \caption{Individualized absolute benefit predictions based on baseline risk when using a constant treatment effect approach, a linear interaction approach and RCS smoothing using 3,4 and 5 knots. Risk stratified estimates of absolute benefit are presented within quartiles of baseline risk as reference.}\label{fig:gusto}
\end{figure}

\hypertarget{discussion}{%
\section{Discussion}\label{discussion}}

The linear interaction model and the RCS-3 model both displayed very
good performance under many of the considered simulation scenarios, in
contrast with the constant relative treatment effect model. The linear
interaction model was optimal in cases with smaller sample sizes and
moderately performing baseline risk prediction models, that is, it had
lower RMSE, was better calibrated for benefit and had better
discrimination for benefit, even in scenarios with strong quadratic
deviations. In scenarios with true non-monotonic deviations, the linear
interaction model was outperformed by RCS-3, especially in the presence
of true treatment-related harms. Increasing the sample size or the
prediction model's discriminative ability favored RCS-3, especially in
scenarios with non-monotonic deviations and in the presence of
treatment-related harms.

RCS-4 and RCS-5 proved to be too flexible in all considered scenarios,
as indicated by higher RMSE, increased variability of discrimination for
benefit and worse calibration of benefit predictions. Even with larger
sample sizes and strong quadratic or non-monotonic deviations from the
base case scenario of constant relative treatment effects, these more
flexible restricted cubic splines did not outperform the simpler RCS-3
These approaches may only be helpful if we expect more extreme patterns
of heterogeneous treatment effects compared to the quadratic deviations
considered here.

The constant treatment effect model, despite having adequate performance
in the presence of weak treatment effect heterogeneity on the relative
scale, quickly broke down with stronger deviations from constant
relative treatment effects. In these cases, the stratified approach
generally had lower error rates compared to the constant treatment
effect model. Stepwise treatment benefit estimates are very useful for
demonstrating treatment effect heterogeneity--because estimating
treatment effect requires groups of patients rather than individual
patients--but are not helpful for making individualized absolute benefit
predictions.

Increasing the discriminative ability of the risk model--by increasing
the predictor coefficients of the true risk model--reduced RMSE for all
methods. This increase in discriminative ability translates in higher
variability of predicted risks, which, in turn, allows the considered
methods to better capture absolute treatment benefits. As a consequence,
the increase in discriminative ability of the risk model also led to
higher values of c-for-benefit. Even though risk model performance is
very important for the ability of risk-based methods to predict
treatment benefit, prediction model development was outside the scope of
this work and has already been studied extensively {[}5,13,14{]}.

The adaptive approach had adequate performance, following closely on
average the performance of the ``true'' model in most scenarios.
However, with smaller sample sizes it tended to ``miss'' the
treatment-risk interactions and selected simpler models (Supplementary
Table S7). This resulted in increased RMSE variability in these
scenarios, especially in the case of true strong linear or non-monotonic
deviations from the base case scenario. Therefore, in the case of
smaller sample sizes the simpler linear interaction model is a safer
choice for predicting absolute benefits.

Risk-based approaches to predictive HTE estimate treatment benefit as a
function of baseline risk. A limitation of our study is that we assumed
treatment benefit to be a function of baseline risk in the majority of
the simulation scenarios. We attempted to address that by introducing
constant moderate and strong treatment-related harms, applied on the
absolute scale. Also, we considered a small number of scenarios with
true treatment-covariate interactions, in which our main conclusions
remained the same (Supplement, XX). Future simulation studies could
explore the effect of more extensive deviations from risk-based
treatment effects.

Recent years have seen an increased interest in predictive HTE
approaches focusing on individualized benefit predictions. In our
simulations we only focused on risk-based methods, using baseline risk
as a reference in a two-stage approach to individualizing benefit
predictions. However, there is a plethora of different methods, ranging
from treatment effect modeling to tree-based approaches available in
more recent literature {[}15--17{]}. Simulations are also needed to
assess relative performance and define the settings where these break
down or outperform each other.

In conclusion, the linear interaction approach is a viable option with
smaller sample sizes and/or moderately performing risk prediction
modelsRCS-3 is a better option when non-monotonic deviations from a
constant relative treatment effect and/or substantial treatment-related
harms are anticipated. With larger sample size, an adaptive approach --
selecting the method with optimal AIC -- can also be considered as an
automated alternative.

\newpage

\hypertarget{references}{%
\section{References}\label{references}}

\setlength{\parindent}{-0.25in}
\setlength{\leftskip}{0.25in}

\noindent

\hypertarget{refs}{}
\begin{cslreferences}
\leavevmode\hypertarget{ref-Varadhan2013}{}%
{[}1{]} Varadhan R, Segal JB, Boyd CM, Wu AW, Weiss CO. A framework for
the analysis of heterogeneity of treatment effect in~patient-centered
outcomes research. Journal of Clinical Epidemiology 2013;66:818--25.
\url{https://doi.org/10.1016/j.jclinepi.2013.02.009}.

\leavevmode\hypertarget{ref-Rekkas2020}{}%
{[}2{]} Rekkas A, Paulus JK, Raman G, Wong JB, Steyerberg EW, Rijnbeek
PR, et al. Predictive approaches to heterogeneous treatment effects: A
scoping review. BMC Medical Research Methodology 2020;20.
\url{https://doi.org/10.1186/s12874-020-01145-1}.

\leavevmode\hypertarget{ref-Kent2019}{}%
{[}3{]} Kent DM, Paulus JK, Klaveren D van, D'Agostino R, Goodman S,
Hayward R, et al. The predictive approaches to treatment effect
heterogeneity (PATH) statement. Annals of Internal Medicine 2019;172:35.
\url{https://doi.org/10.7326/m18-3667}.

\leavevmode\hypertarget{ref-PathEnE}{}%
{[}4{]} Kent DM, Klaveren D van, Paulus JK, D'Agostino R, Goodman S,
Hayward R, et al. The predictive approaches to treatment effect
heterogeneity (PATH) statement: Explanation and elaboration. Annals of
Internal Medicine 2019;172:W1. \url{https://doi.org/10.7326/m18-3668}.

\leavevmode\hypertarget{ref-vanKlaveren2019}{}%
{[}5{]} Klaveren D van, Balan TA, Steyerberg EW, Kent DM. Models with
interactions overestimated heterogeneity of treatment effects and were
prone to treatment mistargeting. Journal of Clinical Epidemiology
2019;114:72--83. \url{https://doi.org/10.1016/j.jclinepi.2019.05.029}.

\leavevmode\hypertarget{ref-Kent2010}{}%
{[}6{]} Kent DM, Rothwell PM, Ioannidis JP, Altman DG, Hayward RA.
Assessing and reporting heterogeneity in treatment effects in clinical
trials: A proposal. Trials 2010;11.
\url{https://doi.org/10.1186/1745-6215-11-85}.

\leavevmode\hypertarget{ref-Kent2016}{}%
{[}7{]} Kent DM, Nelson J, Dahabreh IJ, Rothwell PM, Altman DG, Hayward
RA. Risk and treatment effect heterogeneity: Re-analysis of individual
participant data from 32 large clinical trials. International Journal of
Epidemiology 2016:dyw118. \url{https://doi.org/10.1093/ije/dyw118}.

\leavevmode\hypertarget{ref-Harrell1988}{}%
{[}8{]} Harrell FE, Lee KL, Pollock BG. Regression models in clinical
studies: Determining relationships between predictors and response. JNCI
Journal of the National Cancer Institute 1988;80:1198--202.
\url{https://doi.org/10.1093/jnci/80.15.1198}.

\leavevmode\hypertarget{ref-vanKlaveren2018}{}%
{[}9{]} Klaveren D van, Steyerberg EW, Serruys PW, Kent DM. The proposed
``concordance-statistic for benefit'' provided a useful metric when
modeling heterogeneous treatment effects. Journal of Clinical
Epidemiology 2018;94:59--68.
\url{https://doi.org/10.1016/j.jclinepi.2017.10.021}.

\leavevmode\hypertarget{ref-Austin2019}{}%
{[}10{]} Austin PC, Steyerberg EW. The integrated calibration index
(ICI) and related metrics for quantifying the calibration of logistic
regression models. Statistics in Medicine 2019;38:4051--65.
\url{https://doi.org/10.1002/sim.8281}.

\leavevmode\hypertarget{ref-Califf1997}{}%
{[}11{]} Califf RM, Woodlief LH, Harrell FE, Lee KL, White HD, Guerci A,
et al. Selection of thrombolytic therapy for individual patients:
Development of a clinical model. American Heart Journal 1997;133:630--9.
\url{https://doi.org/10.1016/s0002-8703(97)70164-9}.

\leavevmode\hypertarget{ref-Steyerberg2000}{}%
{[}12{]} Steyerberg EW, Bossuyt PMM, Lee KL. Clinical trials in acute
myocardial infarction: Should we adjust for baseline characteristics?
American Heart Journal 2000;139:745--51.
\url{https://doi.org/10.1016/s0002-8703(00)90001-2}.

\leavevmode\hypertarget{ref-Burke2014}{}%
{[}13{]} Burke JF, Hayward RA, Nelson JP, Kent DM. Using internally
developed risk models to assess heterogeneity in treatment effects in
clinical trials. Circulation: Cardiovascular Quality and Outcomes
2014;7:163--9. \url{https://doi.org/10.1161/circoutcomes.113.000497}.

\leavevmode\hypertarget{ref-Abadie2018}{}%
{[}14{]} Abadie A, Chingos MM, West MR. Endogenous stratification in
randomized experiments. The Review of Economics and Statistics
2018;100:567--80. \url{https://doi.org/10.1162/rest_a_00732}.

\leavevmode\hypertarget{ref-Athey2019}{}%
{[}15{]} Athey S, Tibshirani J, Wager S. Generalized random forests. The
Annals of Statistics 2019;47. \url{https://doi.org/10.1214/18-aos1709}.

\leavevmode\hypertarget{ref-Lu2018}{}%
{[}16{]} Lu M, Sadiq S, Feaster DJ, Ishwaran H. Estimating individual
treatment effect in observational data using random forest methods.
Journal of Computational and Graphical Statistics 2018;27:209--19.
\url{https://doi.org/10.1080/10618600.2017.1356325}.

\leavevmode\hypertarget{ref-Wager2018}{}%
{[}17{]} Wager S, Athey S. Estimation and inference of heterogeneous
treatment effects using random forests. Journal of the American
Statistical Association 2018;113:1228--42.
\url{https://doi.org/10.1080/01621459.2017.1319839}.
\end{cslreferences}

\setlength{\parindent}{0in}
\setlength{\leftskip}{0in}

\noindent


\end{document}
