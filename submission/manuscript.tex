\documentclass{article}

\usepackage{arxiv}

\usepackage[utf8]{inputenc} % allow utf-8 input
\usepackage[T1]{fontenc}    % use 8-bit T1 fonts
\usepackage{lmodern}        % https://github.com/rstudio/rticles/issues/343
\usepackage{hyperref}       % hyperlinks
\usepackage{url}            % simple URL typesetting
\usepackage{booktabs}       % professional-quality tables
\usepackage{amsfonts}       % blackboard math symbols
\usepackage{nicefrac}       % compact symbols for 1/2, etc.
\usepackage{microtype}      % microtypography
\usepackage{lipsum}
\usepackage{graphicx}

\title{Linear interaction of treatment with baseline risk was sufficient
for predicting individualized benefit}

\author{
    Alexandros Rekkas
   \\
    Department of Medical Informatics \\
    Erasmus Medical Center \\
  Rotterdam, The Netherlands \\
  \texttt{} \\
   \And
    Peter R. Rijnbeek
   \\
    Department of Medical Informatics \\
    Erasmus Medical Center \\
  Rotterdam, The Netherlands \\
  \texttt{} \\
   \And
    Ewout W. Steyerberg
   \\
    Department of Biomedical Data Sciences \\
    Leiden University Medical Center \\
  Leiden, The Netherlands \\
  \texttt{} \\
   \And
    David van Klaveren
   \\
    Department of Public Health \\
    Erasmus Medical Center \\
  Rotterdam, The Netherlands \\
  \texttt{} \\
  }


% Pandoc citation processing
\newlength{\csllabelwidth}
\setlength{\csllabelwidth}{3em}
\newlength{\cslhangindent}
\setlength{\cslhangindent}{1.5em}
% for Pandoc 2.8 to 2.10.1
\newenvironment{cslreferences}%
  {}%
  {\par}
% For Pandoc 2.11+
\newenvironment{CSLReferences}[2] % #1 hanging-ident, #2 entry spacing
 {% don't indent paragraphs
  \setlength{\parindent}{0pt}
  % turn on hanging indent if param 1 is 1
  \ifodd #1 \everypar{\setlength{\hangindent}{\cslhangindent}}\ignorespaces\fi
  % set entry spacing
  \ifnum #2 > 0
  \setlength{\parskip}{#2\baselineskip}
  \fi
 }%
 {}
\usepackage{calc} % for calculating minipage widths
\newcommand{\CSLBlock}[1]{#1\hfill\break}
\newcommand{\CSLLeftMargin}[1]{\parbox[t]{\csllabelwidth}{#1}}
\newcommand{\CSLRightInline}[1]{\parbox[t]{\linewidth - \csllabelwidth}{#1}\break}
\newcommand{\CSLIndent}[1]{\hspace{\cslhangindent}#1}

\renewcommand*\familydefault{\sfdefault}
\usepackage{setspace}
\usepackage{amsmath}
\doublespacing
\usepackage[pagewise, left]{lineno}
\usepackage{amssymb}
\usepackage{bm}
\usepackage{caption}
\usepackage{booktabs}
\date{}
\newcommand\given[1][]{\:#1\vert\:}
\usepackage{booktabs}
\usepackage{longtable}
\usepackage{array}
\usepackage{multirow}
\usepackage{wrapfig}
\usepackage{float}
\usepackage{colortbl}
\usepackage{pdflscape}
\usepackage{tabu}
\usepackage{threeparttable}
\usepackage{threeparttablex}
\usepackage[normalem]{ulem}
\usepackage{makecell}
\usepackage{xcolor}


\begin{document}
\maketitle

\def\tightlist{}


\begin{abstract}
\textbf{Objective:} To compare different risk-based methods predicting
individualized treatment effects in RCTs. \textbf{Study Design and
Setting:} We simulated data using diverse assumptions for a baseline
prognostic index of risk (PI), the shape of its interaction with
treatment (none, linear, quadratic or non-monotonic) and the magnitude
of treatment-related harms. In each sample we predicted absolute benefit
using: models with a constant relative treatment effect; stratification
in quarters of the PI; models including a linear interaction of
treatment with the PI; models including an interaction of treatment with
a restricted cubic spline (RCS) transformation of the PI; an adaptive
approach using Akaike's Information Criterion. We evaluated predictive
performance using root mean squared error and measures of discrimination
and calibration for benefit. Starting from a base scenario (sample size
4,250, treatment odds ratio 0.8, AUC of the PI 0.75), we varied the
sample size, the treatment effect size, and the PI's AUC.
\textbf{Results:} Models including a PI by treatment interaction
performed best with smaller sample sizes and/or lower AUC of the PI.
Otherwise, RCS (3 knots) models proved more robust for the majority of
the deviation settings. The adaptive approach was unstable with smaller
sample sizes. \textbf{Conclusion:} Depending on the setting, the linear
interaction or the RCS (3 knots) model should be preferred.
\end{abstract}

\keywords{
    treatment effect heterogeneity
   \and
    absolute benefit
   \and
    prediction models
  }

\doublespacing 
\linenumbers

\hypertarget{introduction}{%
\section{Introduction}\label{introduction}}

Predictive approaches for assessing heterogeneity of treatment effects
(HTE) aim at the development of models predicting either individualized
effects or which of two (or more) treatments is better for an individual
{[}1{]}. In prior work, we divided such methods in three broader
categories based on the reference class used for defining patient
similarity when making individualized predictions or recommendations
{[}2{]}. Risk-modeling approaches use prediction of baseline risk as the
reference; treatment effect modeling approaches also model
treatment-covariate interactions, in addition to risk factors; optimal
treatment regime approaches focus on developing treatment assignment
rules and therefore rely heavily on modeling treatment effect modifiers.

Risk-modeling approaches to predictive HTE analyses provide a viable
option in the absence of well-established treatment effect modifiers
{[}3,4{]}. In simulations, modeling of effect modifiers, i.e.
treatment-covariate interactions, often led to miscalibrated predictions
of benefit, while risk-based methods proved quite robust {[}5{]}. Most
often, risk-modeling approaches are carried out in two steps: first a
risk prediction model is developed externally or internally on the
entire RCT population, ``blinded'' to treatment; then the RCT population
is stratified using this prediction model to evaluate risk-based
treatment effect variation {[}6{]}. However, even though estimates at
the risk subgroup level may be accurate, these estimates do not apply to
individual patients, especially for patients with predicted risk at the
boundaries of the risk intervals. Hence, the risk-stratified approach is
useful for exploring and presenting HTE, but is not useful for
supporting treatment decisions for individual patients.

To individualize treatment effects, the recent PATH statement suggested
various risk-based models including a prognostic index of baseline risk
(PI) and treatment assignment {[}3,4{]}. We aimed to summarize and
compare different risk-based models for predicting individualized
treatment effects. We simulated RCT settings to compare the performance
of these models under different assumptions of the relationship between
baseline risk and treatment. We illustrated the different models by a
case study of predicting individualized effects of tissue plasminogen
activator (tPA) versus streptokinase treatment in patients with an acute
myocardial infarction (MI).

\hypertarget{methods}{%
\section{Methods}\label{methods}}

\hypertarget{simulation-scenarios}{%
\subsection{Simulation scenarios}\label{simulation-scenarios}}

For each patient we generated 8 baseline covariates
\(x_1,\dots,x_4\sim N(0, 1)\) and \(x_5,\dots,x_8\sim B(1, 0.2)\).
Treatment was allocated using a 50:50 split. Outcomes for patients in
the control arm were generated from a logistic regression model
including all baseline covariates. In the base scenarios coefficient
values were such, that the AUC of the logistic regression model was
\(0.75\) and the event rate in the control arm was \(20\%\). Binary
outcomes in the control arm were generated from Bernoulli variables with
true probabilities
\(P(y=1|X, t_x = 0) = \text{expit}(PI)=\frac{e^{PI}}{1+e^{PI}}\).

Outcomes in the treatment arm were generated using 3 base scenarios:
absent treatment effect (OR = 1), moderate treatment effect (OR = 0.8)
and high treatment effect (OR = 0.5). We started with simulating
outcomes based on true constant relative treatment effects for the 3
base scenarios. We then simulated linear, quadratic and non-monotonic
deviations from constant treatment effects using:
\[lp_1 = \gamma_2(PI-c)^2 + \gamma_1(PI-c) + \gamma_0, \] where \(lp_1\)
is the true linear predictor in the treatment arm, so that
\(P(y=1|X, t_x=1) = \text{expit}(lp_1)\). Finally, we simulated
scenarios where a constant absolute harm is applied across all treated
patients. In this case we have
\(P(y=1|X,t_x=1) = \text{expit}(lp_1) + \text{harm}\).

The sample size for the base scenarios was set to 4,250 (\(80\%\) power
for the detection of a marginal OR of 0.8). We evaluated the effect of
smaller or larger sample sizes of 1,063 and 17,000, respectively. We
also evaluated the effect of worse or better discriminative ability for
risk, adjusting the baseline covariate coefficients, such that the AUC
of the regression model in the control arm was 0.65 and 0.85
respectively.

Combining all these settings resulted in a simulation study of 648
scenarios (exact settings in the supplementary material).

\hypertarget{individualized-risk-based-benefit-predictions}{%
\subsection{Individualized risk-based benefit
predictions}\label{individualized-risk-based-benefit-predictions}}

All methods assume that a risk prediction model is available to assign
risk predictions to individual patients. For the simulations we
developed a prediction model internally, using logistic regression
including main effects for all baseline covariates and treatment
assignment. Risk predictions for individual patients were based on
treatment assignment to the control arm, that is setting treatment
assignment to 0.

A \emph{stratified HTE method} has been suggested as an alternative to
traditional subgroup analyses. Patients are stratified into
equally-sized risk strata---in this case based on risk quartiles.
Absolute treatment effects within risk strata are estimated by the
difference in event rate between patients in the control arm and
patients in the treated arm. We considered this approach as a reference,
expecting it to perform worse than the other candidates, as its
objective is not to individualize benefit prediction.

Second, we considered a model which assumes \emph{constant relative
treatment effect} (constant odds ratio). Hence, absolute benefit is
predicted from
\(\hat{\tau}(\bm{x}) = \text{expit}(PI +\log(\text{OR}))\).

Third, we considered a logistic regression model including treatment,
the prognostic index, and their linear interaction. Absolute benefit is
then estimated from
\(\hat{\tau}(\bm{x})=\text{expit}(\beta_0+\beta_{PI}PI) - \text{expit}(\beta_0+\beta_{t_x}+(\beta_{PI}+\beta_*)PI)\).
We will refer to this method as the \emph{linear interaction} approach.

Fourth, we used \emph{restricted cubic splines} (RCS) to relax the
linearity assumption on the effect of the linear predictor {[}7{]}. We
considered splines with 3 (RCS-3), 4 (RCS-4) and 5 (RCS-5) knots to
compare models with different levels of flexibility.

Finally, we considered an adaptive approach using Akaike's Information
Criterion (AIC) for model selection. The candidate models were: a
constant treatment effect model, a model with a linear interaction with
treatment and RCS models with 3, 4 and 5 knots.

\hypertarget{evaluation-metrics}{%
\subsection{Evaluation metrics}\label{evaluation-metrics}}

We evaluated the predictive accuracy of the considered methods by the
root mean squared error (RMSE):

\[\text{RMSE}=\sqrt{\frac{1}{n}\sum_{i=1}^n\big(\tau(\bm{x}_i) - \hat{\tau}(\bm{x}_i)\big)^2}\]
We compared the discriminative ability of the methods under study using
c-for-benefit {[}8{]}. The c-for-benefit represents the probability that
from two randomly chosen matched patient pairs with unequal observed
benefit, the pair with greater observed benefit also has a higher
predicted benefit. To be able to calculate observed benefit, patients in
each treatment arm are ranked based on their predicted benefit and then
matched 1:1 across treatment arms. \emph{Observed} treatment benefit is
defined as the difference of observed outcomes between the untreated and
the treated patient of each matched patient pair. \emph{Predicted}
benefit is defined as the average of predicted benefit within each
matched patient pair.

We evaluated calibration in a similar manner, using the integrated
calibration index (ICI) for benefit {[}9{]}. The observed benefits are
regressed on the predicted benefits using a locally weighted scatterplot
smoother (loess). The ICI-for-benefit is the average absolute difference
between predicted and smooth observed benefit. Values closer to \(0\)
represent better calibration.

\hypertarget{results}{%
\section{Results}\label{results}}

\hypertarget{simulations}{%
\subsection{Simulations}\label{simulations}}

The linear interaction model outperformed all RCS methods in terms of
RMSE in scenarios with true constant relative treatment effect (OR =
0.8, N = 4,250 and AUC = 0.75), strong linear and even strong quadratic
deviations (Figure \ref{fig:rmsebase}). However, with non-monotonic
deviations errors of the linear interaction model increased
substantially, especially in the presence of treatment-related harms. In
these scenarios, RCS-3 proved to be quite robust outperforming all other
methods. The constant treatment effect approach had overall best
performance under true constant treatment effect settings, but was
sensitive to all considered deviations, resulting in increased error
rates. Finally, the adaptive approach had comparable performance to the
best-performing method in each scenario. However, we observed increased
error variability in the case of linear and non-monotonic deviations,
especially for moderate or strong treatment-related harms.

In the case of true constant relative treatment effects, the adaptive
approach selected the constnant effect model the majority of the time,
even in the presence of treatment-related harms (96\% of the time with
no harms to 88\% with strong harms). With strong linear or quadratic
deviations and increasing treatment-related harms the adaptive approach
increasingly favored the linear interaction model. In the case of true
non-monotonic deviations, we observed increasing selection frequencies
for RCS-3 with increasing treatment harms (from 32\% with no harms to
53\% with strong harms), whereas selection frequencies for the linear
interaction model dropped from 40\% with no harms to 7\% with strong
harms. Finally, in the case of strong linear and non-monotonic
deviations selection frequencies of the constant treatment effect model
remained high despite increasing treatment harms (Supplement, Table XX).

\begin{figure}
\includegraphics[width=1\linewidth]{manuscript_files/figure-latex/rmsebase-1} \caption{RMSE of the considered methods across 500 replications calculated in a simulated super-population of size 500,000. The scenario with true constant relative treatment effect had a true prediction AUC of 0.75 and sample size of 4250. Results are presented under moderate linear, strong linear, and strong quadratic deviations from constant relative treatment effects.}\label{fig:rmsebase}
\end{figure}

Increasing the sample size to 17,000 favored RCS-3 the most, as it
achieved lowest or close to lowest RMSE across all scenarios (Figure
\ref{fig:rmsesamplesize}). Especially in cases of strong quadratic and
non-monotonic deviations RCS-3 had lower error rates (median RMSE 0.011
for strong quadratic deviations and 0.010 for non-monotonic deviations
with no treatment-related harms) compared to the linear interaction
approach (median RMSE 0.013 and 0.014, respectively), regardless of
treatment-related harms strength. The issues with the large error
variability of the adaptive approach improved with larger sample sizes.

The adaptive approach tended to increasingly favor smoother methods
(especially RCS-3) with increasing treatment-related harms (see
Supplement, Table XX). However, in the case of true strong quadratic
deviations the opposite was observed: selection frequency of the linear
interaction model increased from 31\% (no harms) to 50\% (strong harms)
whereas for RCS-3 decreased from 52\% (no harms) to 34\% (strong harms).

\begin{figure}
\includegraphics[width=1\linewidth]{manuscript_files/figure-latex/rmsesamplesize-1} \caption{RMSE of the considered methods across 500 replications calculated in a simulated sample of size 500,000. Sample size 17,000 rather than 4250 in Figure \ref{fig:rmsebase}}\label{fig:rmsesamplesize}
\end{figure}

When we increased the AUC of the true prediction model to 0.85 (OR = 0.8
and N = 4,250) the constant effect model outperformed all other methods
in the case of true constant treatment effects, but proved to be the
least robust to deviations. Again, RCS-3 had the lowest error rates in
the case of strong quadratic or monotonic deviations and very comparable
performance to the best-performing linear interaction model in the case
of strong linear deviations (median RMSE 0.016 for RCS-3 compared to
0.014 for the linear interaction model). The adaptive approach, though
it performed similar to the best performing method in each scenario, on
average, had increased variability in error rates in the case of strong
linear and non-monotonic deviations. In these scenarios, the adaptive
approach often selected the constant treatment effect model (23\% and
25\% in the strong linear and non-monotonic deviation scenarios without
treatment-related harms, respectively).

\begin{figure}
\includegraphics[width=1\linewidth]{manuscript_files/figure-latex/rmseauc-1} \caption{RMSE of the considered methods across 500 replications calculated in a simulated sample of size 500,000. True prediction AUC of 0.85 and sample size of 4,250.}\label{fig:rmseauc}
\end{figure}

In terms of discrimination for benefit (OR = 0.8, N = 4,250 and AUC =
0.75), all methods performed similarly, on average (Figure
\ref{fig:discrimination}). In the case of non-monotonic deviations, the
constant effect model had much lower discriminative performance compared
to the rest of the methods (median AUC of 0.04 for the constant effects
model compared to the best-performing RCS-3 with 0.02). The linear
interaction model was the most stable compared to the other methods in
terms of error variability. With increasing number of RCS knots, we
observed decreasing median values and increasing variability of the
c-for-benefit in all scenarios.

\begin{figure}
\includegraphics[width=1\linewidth]{manuscript_files/figure-latex/discrimination-1} \caption{Discrimination for benefit of the considered methods across 500 replications calculated in a simulated sample of size 500,000. True prediction AUC of 0.75 and sample size of 4,250.}\label{fig:discrimination}
\end{figure}

In terms of calibration for benefit, the constant effects model
outperformed all other models in the case of true constant treatment
effects, but was miscalibrated for all deviation scenarios (Figure
\ref{fig:calibration}). The linear interaction model showed best or
close to best calibration across all scenarios and only showed worse
calibration compared to RCS-3 in the case of non-monotonic deviations
and treatment-related harms. The adaptive approach was worse calibrated
in scenarios with strong linear and non-monotonic deviations compared to
the linear interaction model and RCS-3.

The results from all individual scenarios can be explored online at
\url{https://arekkas.shinyapps.io/simulation_viewer/}.

\begin{figure}
\includegraphics[width=1\linewidth]{manuscript_files/figure-latex/calibration-1} \caption{Calibration for benefit of the considered methods across 500 replications calculated in a simulated sample of size 500,000. True prediction AUC of 0.75 and sample size of 4,250.}\label{fig:calibration}
\end{figure}

\hypertarget{case-study}{%
\subsection{Case study}\label{case-study}}

We demonstrate the different methods for individualizing treatment
benefits using data from 30,510 patients with an acute myocardial
infarction (MI) included in the GUSTO-I trial. 10,348 patients were
randomized to tissue plasminogen activator (tPA) treatment and 20,162
were randomized to streptokinase. The outcome of interest was 30-day
mortality, recorded for all patients.

In line with previous analyses {[}10,11{]}, we fitted a logistic
regression model with 6 baseline covariates, i.e.~age, Killip class,
systolic blood pressure, heart rate, an indicator of previous MI, and
the location of MI, to predict 30-day mortality risk. A constant effect
of treatment was included in the model. When deriving risk predictions
for individuals we set the treatment indicator to 0. More information on
model development can be found in the supplement.

We used the risk linear predictor to fit the proposed methods under
study for individualizing absolute benefit predictions. All methods
predicted increasing benefits for patients with higher baseline risk
predictions, but the fitted patterns were clearly different. The
adaptive approach selected the model with RCS smoothing with 4 knots.
However, for very low baseline risk this model predicted decreasing
benefit with increasing risk may be somewhat too flexible. The more
robust models, the linear interaction model or the model with RCS
smoothing (3 knots), gave very similar benefit predictions, followed the
evolution of the stratified estimates very closely and may therefore be
preferable for use in clinical practice. The linear interaction model
had somewhat lower AIC compared to the model with RCS smoothing (3
knots), slightly better cross-validated discrimination (c-for-benefit
0.526 vs 0.525) and quite similar cross-validated calibration (ICI-for
benefit 0.0115 vs 0.0117).

\begin{figure}
\includegraphics[width=1\linewidth]{/home/arekkas/Documents/Projects/arekkas_HteSimulation_XXXX_2021/figures/gusto} \caption{Individualized absolute benefit predictions based on baseline risk when using a constant treatment effect approach, a linear interaction approach and RCS smoothing using 3,4 and 5 knots. Risk stratified estimates of absolute benefit are presented within quartiles of baseline risk as reference.}\label{fig:gusto}
\end{figure}

\hypertarget{discussion}{%
\section{Discussion}\label{discussion}}

The linear interaction model displayed very good performance overall
under many of the considered simulation scenarios. Especially in cases
with smaller sample sizes and moderately performing baseline risk
prediction models it had lower RMSE, was better calibrated for benefit
and had better discrimination for benefit, even in scenarios with strong
quadratic deviations. However, in scenarios with true non-monotonic
deviations, the linear interaction model was outperformed by RCS-3,
especially in the presence of true treatment-related harms. Increasing
the sample size or the prediction model's discrimination favored RCS-3
which had better or very comparable performance to the linear
interaction model, but was more robust to non-monotonic deviations and
the presence of treatment-related harms.

RCS-4 and RCS-5 proved to be too flexible, as indicated by higher RMSE,
increased variability of discrimination for benefit and worse
calibration of benefit predictions. Even with larger sample sizes and
strong quadratic or non-monotonic deviations from the base case scenario
of constant relative treatment effects, these more flexible restricted
cubic splines did not outperform the simpler RCS-3 These approaches may
only be helpful if we expect more extreme patterns of heterogeneous
treatment effects compared to the quadratic deviations considered here.

The constant treatment effect model, despite having adequate performance
in the presence of weak treatment effect heterogeneity on the relative
scale, quickly broke down with stronger deviations from constant
relative treatment effects. In these cases, the stratified approach
generally had lower error rates compared to the constant treatment
effect model. Stepwise treatment benefit estimates are very useful for
demonstrating treatment effect heterogeneity--because estimating
treatment effect requires groups of patients rather than individual
patients--but are not helpful for making individualized absolute benefit
predictions.

Increasing the discriminative ability of the risk model--by increasing
the predictor coefficients of the true risk model--reduced RMSE for all
methods. This increase in discriminative ability translates in higher
variability of predicted risks, which, in turn, allows the considered
methods to better capture absolute treatment benefits. As a consequence,
the increase in discriminative ability of the risk model also led to
higher values of c-for-benefit. Even though risk model performance is
very important for the ability of risk-based methods to predict
treatment benefit, prediction model development was outside the scope of
this work and has already been studied extensively {[}5,12,13{]}.

The adaptive approach had adequate performance, following closely on
average the performance of the ``true'' model in most scenarios.
However, with smaller sample sizes it tended to ``miss'' the
treatment-risk interactions and selected simpler models (Supplementary
Table S7). This resulted in increased RMSE variability in these
scenarios, especially in the case of true strong linear or non-monotonic
deviations from the base case scenario. Therefore, in the case of
smaller sample sizes the simpler linear interaction model is a safer
choice for predicting absolute benefits.

Risk-based approaches to predictive HTE estimate treatment benefit as a
function of baseline risk. A limitation of our study is that we assumed
treatment benefit to be a function of baseline risk in the majority of
the simulation scenarios. We attempted to address that by introducing
constant moderate and strong treatment-related harms, applied on the
absolute scale. Also, we considered a small number of scenarios with
true treatment-covariate interactions, in which our main conclusions
remained the same (Supplement, XX). Future simulation studies could
explore the effect of more extensive deviations from risk-based
treatment effects.

Recent years have seen an increased interest in predictive HTE
approaches focusing on individualized benefit predictions. In our
simulations we only focused on risk-based methods, using baseline risk
as a reference in a two-stage approach to individualizing benefit
predictions. However, there is a plethora of different methods, ranging
from treatment effect modeling to tree-based approaches available in
more recent literature {[}14--16{]}. Simulations are also needed to
assess relative performance and define the settings where these break
down or outperform each other.

In conclusion, the best option for predicting individualized treatment
benefit using a risk-based approach depends on the setting. With smaller
sample sizes and/or moderately performing risk prediction models the
linear interaction approach is a viable option. When those constraints
are not present or when we anticipate non-negligible treatment-related
harms, RCS-3 is a better option in terms of error rates, discrimination
and calibration for benefit. With larger sample size, an adaptive
approach based on AIC can also be considered as a more automated
alternative.

\newpage

\hypertarget{references}{%
\section{References}\label{references}}

\setlength{\parindent}{-0.25in}
\setlength{\leftskip}{0.25in}

\noindent

\hypertarget{refs}{}
\begin{cslreferences}
\leavevmode\hypertarget{ref-Varadhan2013}{}%
{[}1{]} Varadhan R, Segal JB, Boyd CM, Wu AW, Weiss CO. A framework for
the analysis of heterogeneity of treatment effect in~patient-centered
outcomes research. Journal of Clinical Epidemiology 2013;66:818--25.
\url{https://doi.org/10.1016/j.jclinepi.2013.02.009}.

\leavevmode\hypertarget{ref-Rekkas2020}{}%
{[}2{]} Rekkas A, Paulus JK, Raman G, Wong JB, Steyerberg EW, Rijnbeek
PR, et al. Predictive approaches to heterogeneous treatment effects: A
scoping review. BMC Medical Research Methodology 2020;20.
\url{https://doi.org/10.1186/s12874-020-01145-1}.

\leavevmode\hypertarget{ref-Kent2019}{}%
{[}3{]} Kent DM, Paulus JK, Klaveren D van, D'Agostino R, Goodman S,
Hayward R, et al. The predictive approaches to treatment effect
heterogeneity (PATH) statement. Annals of Internal Medicine 2019;172:35.
\url{https://doi.org/10.7326/m18-3667}.

\leavevmode\hypertarget{ref-PathEnE}{}%
{[}4{]} Kent DM, Klaveren D van, Paulus JK, D'Agostino R, Goodman S,
Hayward R, et al. The predictive approaches to treatment effect
heterogeneity (PATH) statement: Explanation and elaboration. Annals of
Internal Medicine 2019;172:W1. \url{https://doi.org/10.7326/m18-3668}.

\leavevmode\hypertarget{ref-vanKlaveren2019}{}%
{[}5{]} Klaveren D van, Balan TA, Steyerberg EW, Kent DM. Models with
interactions overestimated heterogeneity of treatment effects and were
prone to treatment mistargeting. Journal of Clinical Epidemiology
2019;114:72--83. \url{https://doi.org/10.1016/j.jclinepi.2019.05.029}.

\leavevmode\hypertarget{ref-Kent2010}{}%
{[}6{]} Kent DM, Rothwell PM, Ioannidis JP, Altman DG, Hayward RA.
Assessing and reporting heterogeneity in treatment effects in clinical
trials: A proposal. Trials 2010;11.
\url{https://doi.org/10.1186/1745-6215-11-85}.

\leavevmode\hypertarget{ref-Harrell1988}{}%
{[}7{]} Harrell FE, Lee KL, Pollock BG. Regression models in clinical
studies: Determining relationships between predictors and response. JNCI
Journal of the National Cancer Institute 1988;80:1198--202.
\url{https://doi.org/10.1093/jnci/80.15.1198}.

\leavevmode\hypertarget{ref-vanKlaveren2018}{}%
{[}8{]} Klaveren D van, Steyerberg EW, Serruys PW, Kent DM. The proposed
``concordance-statistic for benefit'' provided a useful metric when
modeling heterogeneous treatment effects. Journal of Clinical
Epidemiology 2018;94:59--68.
\url{https://doi.org/10.1016/j.jclinepi.2017.10.021}.

\leavevmode\hypertarget{ref-Austin2019}{}%
{[}9{]} Austin PC, Steyerberg EW. The integrated calibration index (ICI)
and related metrics for quantifying the calibration of logistic
regression models. Statistics in Medicine 2019;38:4051--65.
\url{https://doi.org/10.1002/sim.8281}.

\leavevmode\hypertarget{ref-Califf1997}{}%
{[}10{]} Califf RM, Woodlief LH, Harrell FE, Lee KL, White HD, Guerci A,
et al. Selection of thrombolytic therapy for individual patients:
Development of a clinical model. American Heart Journal 1997;133:630--9.
\url{https://doi.org/10.1016/s0002-8703(97)70164-9}.

\leavevmode\hypertarget{ref-Steyerberg2000}{}%
{[}11{]} Steyerberg EW, Bossuyt PMM, Lee KL. Clinical trials in acute
myocardial infarction: Should we adjust for baseline characteristics?
American Heart Journal 2000;139:745--51.
\url{https://doi.org/10.1016/s0002-8703(00)90001-2}.

\leavevmode\hypertarget{ref-Burke2014}{}%
{[}12{]} Burke JF, Hayward RA, Nelson JP, Kent DM. Using internally
developed risk models to assess heterogeneity in treatment effects in
clinical trials. Circulation: Cardiovascular Quality and Outcomes
2014;7:163--9. \url{https://doi.org/10.1161/circoutcomes.113.000497}.

\leavevmode\hypertarget{ref-Abadie2018}{}%
{[}13{]} Abadie A, Chingos MM, West MR. Endogenous stratification in
randomized experiments. The Review of Economics and Statistics
2018;100:567--80. \url{https://doi.org/10.1162/rest_a_00732}.

\leavevmode\hypertarget{ref-Athey2019}{}%
{[}14{]} Athey S, Tibshirani J, Wager S. Generalized random forests. The
Annals of Statistics 2019;47. \url{https://doi.org/10.1214/18-aos1709}.

\leavevmode\hypertarget{ref-Lu2018}{}%
{[}15{]} Lu M, Sadiq S, Feaster DJ, Ishwaran H. Estimating individual
treatment effect in observational data using random forest methods.
Journal of Computational and Graphical Statistics 2018;27:209--19.
\url{https://doi.org/10.1080/10618600.2017.1356325}.

\leavevmode\hypertarget{ref-Wager2018}{}%
{[}16{]} Wager S, Athey S. Estimation and inference of heterogeneous
treatment effects using random forests. Journal of the American
Statistical Association 2018;113:1228--42.
\url{https://doi.org/10.1080/01621459.2017.1319839}.
\end{cslreferences}

\setlength{\parindent}{0in}
\setlength{\leftskip}{0in}

\noindent

\bibliographystyle{unsrt}
\bibliography{references.bib}


\end{document}
